\sekshun{Statements}
\label{Statements}
\index{statement}

Chapel is an imperative language with statements that may have side
effects.  Statements allow for the sequencing of program execution.
Chapel provides the following statements:

\begin{syntax}
statement:
  block-statement
  expression-statement
  assignment-statement
  swap-statement
  io-statement
  conditional-statement
  select-statement
  while-do-statement
  do-while-statement
  for-statement
  label-statement
  break-statement
  continue-statement
  param-for-statement
  use-statement
  defer-statement
  empty-statement
  return-statement
  yield-statement
  module-declaration-statement
  procedure-declaration-statement
  external-procedure-declaration-statement
  exported-procedure-declaration-statement
  iterator-declaration-statement
  method-declaration-statement
  type-declaration-statement
  variable-declaration-statement
  remote-variable-declaration-statement
  on-statement
  cobegin-statement
  coforall-statement
  begin-statement
  sync-statement
  serial-statement
  atomic-statement
  forall-statement
  delete-statement
\end{syntax}

Individual statements are defined in the remainder of this chapter
and additionally as follows:

\begin{itemize}
\item return \rsec{The_Return_Statement}
\item yield \rsec{The_Yield_Statement}
\item module declaration \rsec{Modules}
\item procedure declaration \rsec{Function_Definitions}
\item external procedure declaration \rsec{Calling_External_Functions}
\item exporting procedure declaration \rsec{Calling_Chapel_Functions}
\item iterator declaration \rsec{Iterator_Function_Definitions}
\item method declaration \rsec{Class_Methods}
\item type declaration \rsec{Types}
\item variable declaration \rsec{Variable_Declarations}
\item remote variable declaration ~\rsec{remote_variable_declarations}
\item \chpl{on} statement \rsec{On}
\item cobegin, coforall, begin, sync, serial and atomic statements
      \rsec{Task_Parallelism_and_Synchronization}
\item forall \rsec{Data_Parallelism}
\item delete \rsec{Class_Delete}
\end{itemize}

\section{Blocks}
\label{Blocks}
\index{block}

A block is a statement or a possibly empty list of statements that
form their own scope.  A block is given by
\begin{syntax}
block-statement:
  { statements[OPT] }

statements:
  statement
  statement statements
\end{syntax}

Variables defined within a block are local
variables~(\rsec{Local_Variables}).

The statements within a block are executed serially unless the block
is in a cobegin statement~(\rsec{Cobegin}).

\section{Expression Statements}
\label{Expression_Statements}
\index{expressions!statement}
\index{expression statement}
\index{statements!expression}
The expression statement evaluates an expression solely for side
effects. The syntax for an expression statement is given by
\begin{syntax}
expression-statement:
  variable-expression ;
  member-access-expression ;
  call-expression ;
  constructor-call-expression ;
  let-expression ;
\end{syntax}

\section{Assignment Statements}
\label{Assignment_Statements}
\index{assignment}
\index{statements!assignment}

An assignment statement assigns the value of an expression to another
expression, for
example, a variable.  Assignment statements are given by

\index{= (see also assignment)@\chpl{=} (see also assignment)}
\index{+=@\chpl{+=}}
\index{-=@\chpl{-=}}
\index{*=@\chpl{*=}}
\index{/=@\chpl{/=}}
\index{\%=@\chpl{\%=}}
\index{**=@\chpl{**=}}
\index{&=@\chpl{&=}}
\index{|=@\chpl{|=}}
\index{^=@\chpl{^=}}
\index{||=@\chpl{||=}}
\index{&&=@\chpl{&&=}}
\index{<<=@\chpl{<<=}}
\index{>>=@\chpl{>>=}}
\index{operators!assignment}
\begin{syntax}
assignment-statement:
  lvalue-expression assignment-operator expression

assignment-operator: one of
   = $ $ $ $ += $ $ $ $ -= $ $ $ $ *= $ $ $ $ /= $ $ $ $ %= $ $ $ $ **= $ $ $ $ &= $ $ $ $ |= $ $ $ $ ^= $ $ $ $ &&= $ $ $ $ ||= $ $ $ $ <<= $ $ $ $ >>=
\end{syntax}

\index{operators!compound assignment}
\index{operators!assignment!compound}
\index{operators!simple assignment}
\index{operators!assignment!simple}
The assignment operators that contain a binary operator symbol as a prefix
are \emph{compound assignment} operators.  The remaining assignment
operator \chpl{=} is called \emph{simple assignment}.

The expression on the left-hand side of the assignment operator must
be a valid lvalue~(\rsec{LValue_Expressions}).  It is evaluated before the
expression on the right-hand side of the assignment operator, which
can be any expression.

When the left-hand side is of a numerical type, there is
an implicit conversion~(\rsec{Implicit_Conversions})
of the right-hand side expression
to the type of the left-hand side expression.  Additionally, for simple
assignment, if the left-hand side is of Boolean type, the right-hand side is
implicitly converted to the type of the left-hand side (i.e. a \chpl{bool(?w)}
with the same width \chpl{w}).

For simple assignment, the validity and semantics of assigning between
classes~(\rsec{Class_Assignment}), records~(\rsec{Record_Assignment}),
unions~(\rsec{Union_Assignment}), tuples~(\rsec{Tuple_Assignment}),
ranges~(\rsec{Range_Assignment}),
domains~(\rsec{Domain_Assignment}), and arrays~(\rsec{Array_Assignment})
are discussed in these later sections.

A compound assignment is
shorthand for applying the binary operator to the left- and
right-hand side expressions and then assigning the result
to the left-hand side expression.
For numerical types, the left-hand side expression is evaluated only once,
and there is an implicit conversion of the result of the binary operator
to the type of the left-hand side expression.  Thus, for
example, \chpl{x += y} is equivalent to \chpl{x = x + y} where the
expression \chpl{x} is evaluated once.

For all other compound assignments, Chapel provides a completely generic
catch-all implementation defined in the obvious way.  For example:

\begin{chapel}
inline proc +=(ref lhs, rhs) {
  lhs = lhs + rhs;
}
\end{chapel}

Thus, compound assignment can be used with operands of arbitrary types,
provided that the following provisions are met: If the type of the left-hand
argument of a compound assignment operator \chpl{op=} is $L$ and that of the
right-hand argument is $R$, then a definition for the corresponding binary
operator \chpl{op} exists, such that $L$ is coercible to the type of its
left-hand formal and $R$ is coercible to the type of its right-hand formal.
Further, the result of \chpl{op} must be coercible to $L$, and there must exist
a definition for simple assignment between objects of type $L$.

Both simple and compound assignment operators can be overloaded for different
types using operator overloading~(\rsec{Function_Overloading}).
In such an overload, the left-hand side expression should have
\chpl{ref} intent and be modified within the body of the function.  The return
type of the function should be \chpl{void}.

\section{The Swap Statement}
\label{The_Swap_Statement}
\index{swap!statement}
\index{statements!swap}
\index{swap!operator}
\index{operators!swap}

% TODO: if appropriate, define swap as a sequence of three assignments
The swap statement indicates to swap the values in the expressions
on either side of the swap operator.  Since both expressions are assigned
to, each must be a valid lvalue expression~(\rsec{LValue_Expressions}).

The swap operator can be overloaded for different types using operator
overloading~(\rsec{Function_Overloading}).
\begin{syntax}
swap-statement:
  lvalue-expression swap-operator lvalue-expression

swap-operator:
  <=>
\end{syntax}

To implement the swap operation, the compiler uses temporary variables
as necessary.

\begin{example}
When resolved to the default swap operator, the following swap statement
\begin{chapel}
var a, b: real;

a <=> b;
\end{chapel}
is semantically equivalent to:
\begin{chapel}
const t = b;
b = a;
a = t;
\end{chapel}
\end{example}

\section{The I/O Statement}
\label{The_IO_Statement}
\index{IO!statement}
\index{statements!IO}
\index{IO!operator}
\index{operators!IO}

The I/O operator indicates writing to the left-hand-side the value in the
right-hand-side; or reading from the left-hand-side and storing the result
in the variable on the right-hand-side. This operator can be chained with
other I/O operator calls.

The I/O operator can be overloaded for different types using operator
overloading~(\rsec{Function_Overloading}).
\begin{syntax}
io-statement:
  io-expression io-operator expression

io-expression:
  expression
  io-expression io-operator expression

io-operator:
  <`(*$\sim$*)'>
\end{syntax}

See the module documentation on I/O for details on how to use the
I/O statement.

\begin{example}
In the example below,
\begin{chapel}
var w = opentmp().writer(); // a channel
var a: real;
var b: int;

w <(*$\sim$*)> a <(*$\sim$*)> b;
\end{chapel}
the I/O operator is left-associative and indicates writing \chpl{a}
and then \chpl{b} to \chpl{w} in this case.
\end{example}


\section{The Conditional Statement}
\label{The_Conditional_Statement}
\index{statements!conditional}
\index{if@\chpl{if}}
\index{then@\chpl{then}}
\index{else@\chpl{else}}
\index{conditional statements}

The conditional statement allows execution to choose between two
statements based on the evaluation of an expression of \chpl{bool}
type. The syntax for a conditional statement is given by
\begin{syntax}
conditional-statement:
  `if' expression `then' statement else-part[OPT]
  `if' expression block-statement else-part[OPT]

else-part:
  `else' statement
\end{syntax}

A conditional statement evaluates an expression of bool type. If the
expression evaluates to true, the first statement in the conditional
statement is executed.  If the expression evaluates to false and the
optional else-clause exists, the statement following the
\chpl{else} keyword is executed.

If the expression is a parameter, the conditional statement is folded
by the compiler. If the expression evaluates to true, the first
statement replaces the conditional statement. If the expression
evaluates to false, the second statement, if it exists, replaces the
conditional statement; if the second statement does not exist, the
conditional statement is removed.

Each statement embedded in the {\em conditional-statement} has its own
scope whether or not an explicit block surrounds it.

\index{conditional statement!dangling else}
If the statement that immediately follows the optional \chpl{then}
keyword is a conditional statement and it is not in a block, the
else-clause is bound to the nearest preceding conditional statement
without an else-clause.
The statement in the else-clause can be a conditional statement, too.

\begin{chapelexample}{conditionals.chpl}
The following function prints \chpl{two} when \chpl{x} is \chpl{2}
and \chpl{B,four} when \chpl{x} is \chpl{4}.
\begin{chapel}
proc condtest(x:int) {
  if x > 3 then
    if x > 5 then
      write("A,");
    else
      write("B,");

  if x == 2 then
    writeln("two");
  else if x == 4 then
    writeln("four");
  else
    writeln("other");
}
\end{chapel}
\begin{chapelpost}
for i in 2..6 do condtest(i);
\end{chapelpost}
\begin{chapeloutput}
two
other
B,four
B,other
A,other
\end{chapeloutput}
\end{chapelexample}

\pagebreak
\section{The Select Statement}
\label{The_Select_Statement}
\index{select@\chpl{select}}
\index{when@\chpl{when}}
\index{otherwise@\chpl{otherwise}}
\index{statements!select@\chpl{select}}
\index{statements!when@\chpl{when}}
\index{statements!otherwise@\chpl{otherwise}}

The select statement is a multi-way variant of the conditional
statement.  The syntax is given by:
\begin{syntax}
select-statement:
  `select' expression { when-statements }

when-statements:
  when-statement
  when-statement when-statements

when-statement:
  `when' expression-list `do' statement
  `when' expression-list block-statement
  `otherwise' statement
  `otherwise' `do' statement

expression-list:
  expression
  expression , expression-list
\end{syntax}
The expression that follows the keyword \chpl{select}, the select
expression, is evaluated once and its value is then compared
with the list of case expressions following each
\chpl{when} keyword. These values are compared using the equality
operator \chpl{==}.  If the expressions cannot be compared with the
equality operator, a compile-time error is generated.  The first case
expression that contains an expression where that comparison
is \chpl{true} will be selected and control transferred to the
associated statement.  If the comparison is always \chpl{false}, the
statement associated with the keyword \chpl{otherwise}, if it exists,
will be selected and control transferred to it.  There may be at most
one \chpl{otherwise} statement and its location within the select
statement does not matter.

Each statement embedded in the {\em when-statement} or the {\em otherwise-statement} has its own scope
whether or not an explicit block surrounds it.

\section{The While Do and Do While Loops}
\label{The_While_and_Do_While_Loops}
\index{while loops}
\index{while@\chpl{while}}
\index{statements!while@\chpl{while}}

There are two variants of the while loop in Chapel.  The syntax of the
while-do loop is given by:
\begin{syntax}
while-do-statement:
  `while' expression `do' statement
  `while' expression block-statement
\end{syntax}
The syntax of the do-while loop is given by:
\begin{syntax}
do-while-statement:
  `do' statement `while' expression ;
\end{syntax}
In both variants, the expression evaluates to a value of type \chpl{bool}
which determines when the loop terminates and control continues with
the statement following the loop.

\pagebreak
The while-do loop is executed as follows:
\begin{enumerate}
\item The expression is evaluated.
\item If the expression evaluates to \chpl{false}, the statement is
  not executed and control continues to the statement following the
  loop.
\item If the expression evaluates to \chpl{true}, the statement is
  executed and control continues to step 1, evaluating the expression
  again.
\end{enumerate}

The do-while loop is executed as follows:
\begin{enumerate}
\item The statement is executed.
\item The expression is evaluated.
\item If the expression evaluates to \chpl{false}, control continues
  to the statement following the loop.
\item If the expression evaluates to \chpl{true}, control continues to
  step 1 and the the statement is executed again.
\end{enumerate}
In this second form of the loop, note that the statement is executed
unconditionally the first time.

\begin{chapelexample}{while.chpl}
The following example illustrates the difference between
the \sntx{do-while-statement} and the \sntx{while-do-statement}.  The body of
the do-while loop is always executed at least once, even if the loop conditional
is already false when it is entered.  The code
\begin{chapel}
var t = 11;

writeln("Scope of do while loop:");
do {
  t += 1;
  writeln(t);
} while (t <= 10);

t = 11;
writeln("Scope of while loop:");
while (t <= 10) {
  t += 1;
  writeln(t);
}
\end{chapel}
produces the output
\begin{chapelprintoutput}{}
Scope of do while loop:
12
Scope of while loop:
\end{chapelprintoutput}
\end{chapelexample}

\pagebreak
Chapel do-while loops differ from those found in most other languages in
one important regard.  If the body of a do-while statement is a block
statement and new variables are defined within that block statement,
then the scope of those variables extends to cover the loop's
termination expression.
\begin{chapelexample}{do-while.chpl}
The following example demonstrates that the scope of the variable t
includes the loop termination expression.
\begin{chapel}
var i = 0;
do {
  var t = i;
  i += 1;
  writeln(t);
} while (t != 5);
\end{chapel}
produces the output
\begin{chapelprintoutput}{}
0
1
2
3
4
5
\end{chapelprintoutput}
\end{chapelexample}


\section{The For Loop}
\label{The_For_Loop}
\index{for loops}
\index{for@\chpl{for}}
\index{statements!for@\chpl{for}}

The for loop iterates over ranges, domains, arrays, iterators, or any
class that implements an iterator named \chpl{these}.  The syntax of
the for loop is given by:
\begin{syntax}
for-statement:
  `for' index-var-declaration `in' iteratable-expression `do' statement
  `for' index-var-declaration `in' iteratable-expression block-statement
  `for' iteratable-expression `do' statement
  `for' iteratable-expression block-statement

index-var-declaration:
  identifier
  tuple-grouped-identifier-list

iteratable-expression:
  expression
  `zip' ( expression-list )
\end{syntax}

The \sntx{index-var-declaration} declares new variables for the scope
of the loop.  It may specify a new identifier or may specify multiple
identifiers grouped using a tuple notation in order to destructure the
values returned by the iterator expression, as described
in~\rsec{Indices_in_a_Tuple}.

The \sntx{index-var-declaration} is optional and may be omitted if the
indices do not need to be referenced in the loop.

If the iteratable-expression begins with the keyword \chpl{zip} followed
by a parenthesized expression-list, the listed expressions must support
zipper iteration.

\subsection{Zipper Iteration}
\label{Zipper_Iteration}
\index{zipper iteration}
\index{iteration!zipper}

When multiple iterators are iterated over in a zipper context, on each
iteration, each expression is iterated over, the values are returned
by the iterators in a tuple and assigned to the index, and then
statement is executed.

The shape of each iterator, the rank and the extents in each
dimension, must be identical.

\begin{chapelexample}{zipper.chpl}
The output of
\begin{chapel}
for (i, j) in zip(1..3, 4..6) do
  write(i, " ", j, " ");
\end{chapel}
\begin{chapelpost}
writeln();
\end{chapelpost}
is
\begin{chapelprintoutput}{}
1 4 2 5 3 6 
\end{chapelprintoutput}
\end{chapelexample}

\subsection{Parameter For Loops}
\label{Parameter_For_Loops}
\index{statements!param for}
\index{for loops!parameters}
\index{for@\chpl{for}}
\index{param@\chpl{param}}

Parameter for loops are unrolled by the compiler so that the index
variable is a parameter rather than a variable.  The syntax for a
parameter for loop statement is given by:
\begin{syntax}
param-for-statement:
  `for' `param' identifier `in' param-iteratable-expression `do' statement
  `for' `param' identifier `in' param-iteratable-expression block-statement

param-iteratable-expression:
  range-literal
  range-literal `by' integer-literal
\end{syntax}
Parameter for loops are restricted to iteration over range literals
with an optional by expression where the bounds and stride must be
parameters.  The loop is then unrolled for each iteration.

\pagebreak
\section{The Break, Continue and Label Statements}
\label{Label_Break_Continue}
\index{statements!jumps}
\index{label@\chpl{label}}
\index{break@\chpl{break}}
\index{continue@\chpl{continue}}
\index{statements!label@\chpl{label}}
\index{statements!break@\chpl{break}}
\index{statements!continue@\chpl{continue}}

The break- and continue-statements are used to alter the flow of control within a
loop construct.  A break-statement causes flow to exit the containing loop and
resume with the statement immediately following it.  A continue-statement causes
control to jump to the end of the body of the containing loop and resume
execution from there.  By default, break- and continue-statements exit
or skip the body of the immediately-containing loop construct.

The label-statement is used to name a specific loop so that \chpl{break}
and \chpl{continue} can exit or resume a less-nested loop.
Labels can only be attached to for-, while-do- and do-while-statements.
When a break statement has a label, execution continues with the first statement
following the loop statement with the matching label.  When a continue statement
has a label, execution continues at the end of the body of the loop with the
matching label.  If there is no containing loop construct with a matching label,
a compile-time error occurs.

The syntax for label, break, and continue statements is given by:
\begin{syntax}
break-statement:
  `break' identifier[OPT] ;

continue-statement:
  `continue' identifier[OPT] ;

label-statement:
  `label' identifier statement
\end{syntax}

Break-statements cannot be used to exit parallel loops.

\begin{rationale}
Breaks are not permitted in parallel loops because the execution order
of the iterations of parallel loops is not defined.
\end{rationale}

\begin{future}
We expect to support a \emph{eureka} concept which would enable one or
more tasks to stop the execution of all current and future iterations
of the loop.
\end{future}

\begin{example}
In the following code, the index of the first element in each row of
\chpl{A} that is equal to \chpl{findVal} is printed.  Once a match is
found, the continue statement is executed causing the outer loop to
move to the next row.
\begin{chapel}
label outer for i in 1..n {
  for j in 1..n {
    if A[i, j] == findVal {
      writeln("index: ", (i, j), " matches.");
      continue outer;
    }
  }
}
\end{chapel}
\end{example}

\section{The Use Statement}
\label{The_Use_Statement}
\index{use@\chpl{use}}
\index{statements!use@\chpl{use}}
\index{modules!using}
\index{types!enumerated!using}

The use statement provides access to the constants in an enumerated type or
the public symbols of a module within the scope in which it appears without
the need to provide a full qualified path.

The syntax of the use statement is given by:

\begin{syntax}
use-statement:
  `use' module-or-enum-name-list ;

module-or-enum-name-list:
  module-or-enum-name limitation-clause[OPT]
  module-or-enum-name , module-or-enum-name-list

module-or-enum-name:
  identifier
  identifier . module-or-enum-name

limitation-clause:
  `except' exclude-list
  `only' rename-list[OPT]

exclude-list:
  identifier-list
  $ * $

rename-list:
  rename-base
  rename-base , rename-list

rename-base:
  identifier `as' identifier
  identifier
\end{syntax}

For example, the program
\begin{chapelexample}{use1.chpl}
\begin{chapel}
module M1 {
  proc foo() {
    writeln("In M1's foo.");
  }
}

module M2 {
  proc main() {
    writeln("In M2's main.");
    M1.foo();
  }
}
\end{chapel}
prints out
\begin{chapelprintoutput}{}
In M2's main.
In M1's foo.
\end{chapelprintoutput}
\end{chapelexample}

This program is equivalent to:
\begin{chapelexample}{use2.chpl}
\begin{chapel}
module M1 {
  proc foo() {
    writeln("In M1's foo.");
  }
}

module M2 {
  proc main() {
    use M1;

    writeln("In M2's main.");
    foo();
  }
}
\end{chapel}

which also prints out
\begin{chapelprintoutput}{}
In M2's main.
In M1's foo.
\end{chapelprintoutput}
\end{chapelexample}

The names that are imported by a use statement are inserted in to a
new scope that immediately encloses the scope within which the
statement appears.  This implies that the position of the use
statement within a scope has no effect on its behavior.  If a scope
includes multiple use statements then the imported names are inserted
in to a common enclosing scope.

An error is signaled if multiple enumeration constants or public module-level
symbols would be inserted into this enclosing scope with the same name, and
that name is referenced by other statements in the same scope as the use.


Use statements are transitive by default: if a module A uses a module
B, and module B contains a use of a module or enumerated type C, then
C's public symbols may also be visible within A.  The exception to
this occurs when B has public symbols that shadow symbols with the
same name in C.

This notion of transitivity extends to the case in which a scope
imports symbols from multiple modules or constants from multiple
enumeration types.  For example if a module A uses modules B1, B2, B3
and modules B1, B2, B3 use modules C1, C2, C3 respectively, then all
of the public symbols in B1, B2, B3 have the potential to shadow the
public symbols of C1, C2, and C3.  However an error is signaled if
C1, C2, C3 have public module level definitions of the same symbol.

% We would like to provide a way to specify that a use should not be
% transitive

An optional \sntx{limitation-clause} may be provided to limit the symbols made
available by the use statement.  If an \chpl{except} list is provided, then all
the visible but unlisted symbols in the module or enumerated type will be made
available without prefix.  If an \chpl{only} list is provided, then just the
listed visible symbols in the module or enumerated type will be made available
without prefix.  All visible symbols not provided via these limited use
statements are still accessible with the module or enumerated type name prefix.
It is an error to provide a name in a \sntx{limitation-clause} that does not
exist or is not visible in the respective module or enumerated type.  If an
\chpl{only} list is left empty or \chpl{except} is followed by $*$ then no
symbols are made available to the scope without prefix.

Within an \chpl{only} list, a visible symbol from that module may optionally be
given a new name using the \chpl{as} keyword.  This new name will be usable from
the scope of the use in place of the old name unless the old name is
additionally specified in the \chpl{only} list.  If a use which renames a symbol
is present at module scope, uses of that module will also be able to reference
that symbol using the new name instead of the old name.  Renaming does not
affect accesses to that symbol via the source module's or enumerated type's
prefix, nor does it affect uses of that module or enumerated type from other
contexts.  It is an error to attempt to rename a symbol that does not exist or
is not visible in the respective module or enumerated type, or to rename a
symbol to a name that is already present in the same \chpl{only} list.  It is,
however, perfectly acceptable to rename a symbol to a name present in the
respective module or enumerated type which was not specified via
that \chpl{only} list.

If a use statement mentions multiple modules or enumerated types or a mix of
these symbols, only the last module or enumerated type can have a
\sntx{limitation-clause}.  Limitation clauses are applied transitively as well
- in the first example, if module A's use of module B contains an \chpl{except}
or \chpl{only} list, that list will also limit which of C's symbols are visible
to A.

For more information on enumerated types, please see~\rsec{Enumerated_Types}.
For use statement rules which are only applicable to modules, please
see~\rsec{Using_Modules}.  For more information on modules in general, please
see~\rsec{Modules}.


\section{The Defer Statement}
\label{The_Defer_Statement}
\index{statements!defer}

A \chpl{defer} statement declares a clean-up action to be run when
exiting a block.
\chpl{defer} is useful because the clean-up action will be run no
matter how the block is exited.

The syntax is:

\begin{syntax}
defer-statement:
  `defer' statement
\end{syntax}

At a given place where control flow exits a block, the in-scope
\chpl{defer} statements and the local variables will be handled in
reverse declaration order. Handling a \chpl{defer} statement consists of
executing the contained clean-up action. Handling a local variable
consists of running its deinitializer if it is of record type.

When an iterator contains a \chpl{defer} statement at the top level, the
associated clean-up action will be executed when the loop running the
iterator exits. \chpl{defer} actions inside a loop body are executed
when that iteration completes.

The following program demonstrates a simple use of \chpl{defer}
to create an action to be executed when returning from a function:

\begin{chapelexample}{defer1.chpl}
\begin{chapel}
class Integer {
  var x:int;
}
proc deferInFunction() {
  var c = new Integer(1);
  writeln("created ", c);
  defer {
    writeln("defer action: deleting ", c);
    delete c;
  }
  // ... (function body, possibly including return statements)
  // The defer action is executed no matter how this function returns.
}
deferInFunction();
\end{chapel}
produces the output
\begin{chapelprintoutput}{}
created {x = 1}
defer action: deleting {x = 1}
\end{chapelprintoutput}
\end{chapelexample}


The following example uses a nested block to demonstrate that \chpl{defer}
is handled when exiting the block in which it is contained:


\begin{chapelexample}{defer2.chpl}
\begin{chapel}
class Integer {
  var x:int;
}
proc deferInNestedBlock() {
  var i = 1;
  writeln("before inner block");
  {
    var c = new Integer(i);
    writeln("created ", c);
    defer {
      writeln("defer action: deleting ", c);
      delete c;
    }
    writeln("in inner block");
    // note, defer action is executed no matter how this block is exited
  }
  writeln("after inner block");
}
deferInNestedBlock();
\end{chapel}
produces the output
\begin{chapelprintoutput}{}
before inner block
created {x = 1}
in inner block
defer action: deleting {x = 1}
after inner block
\end{chapelprintoutput}
\end{chapelexample}

Lastly, this example shows that when \chpl{defer} is used in a loop, the
action will be executed for every loop iteration, whether or not loop
body is exited early.

\begin{chapelexample}{defer3.chpl}
\begin{chapel}
class Integer {
  var x:int;
}
proc deferInLoop() {
  for i in 1..10 {
    var c = new Integer(i);
    writeln("created ", c);
    defer {
      writeln("defer action: deleting ", c);
      delete c;
    }
    writeln(c);
    if i == 2 then
      break;
  }
}
deferInLoop();
\end{chapel}
produces the output
\begin{chapelprintoutput}{}
created {x = 1}
{x = 1}
defer action: deleting {x = 1}
created {x = 2}
{x = 2}
defer action: deleting {x = 2}
\end{chapelprintoutput}
\end{chapelexample}


\section{The Empty Statement}
\label{The_Empty_Statement}
\index{statements!empty}

An empty statement has no effect.  The syntax of an empty statement is
given by
\begin{syntax}
empty-statement:
  ;
\end{syntax}
